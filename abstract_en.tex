\chapter*{Abstract}

The Information Technologies have revolutionized many areas of society. Since many years ago, the health area has been searching for solutions to improve existing processes, having as the main objective the improvement of healthcare services. In this sense, the availability of patient clinical data can be vital to a more effective diagnosis and treatment, by a health professional. This information should be accessible regardless of context, place, time or where it was collected. In order to share this type of data, many countries have initiated projects implementing Electronic Health Record (EHR), including Portugal in 2010.

The EHR complexity and scope makes it an high cost and risk project. In fact, the inherent difficulties of a project like this are as critical as its importance and the impact it has on societies. The implementation of a successful EHR implies, in most cases, the involvement of all stakeholders, from patient to health organizations, along with the professionals themselves, becoming a very long and continuous process.

In the context of this dissertation we pretend to build a architecture proposal for implementing the Portuguese EHR. The proposed architecture should serve as a platform for sharing information between multiple health entities, normalizing processes, applying conventions and chasing interoperability among agents. The ability of the system evolution on one hand, and ease of integration in international projects on the other, are two major concerns to be taken into account throughout the proposal definition process.

This document reflects the study in order to understand what processes and practices are used in implementing these large-scale projects. Accordingly, we present some models of enterprise architectures as well as some architectural styles, both corresponding to high levels of abstraction. Regarding the integration of different systems, this document addresses several international conventions for transmission, encoding and storing of clinical information, such as HL7, openEHR and SNOMED CT. Also, we briefly analyse two International EHR case studies, in UK and Canada. In conclusion it should be noted that regardless of the technical challenges that may arise, the process must always be guided by the objective of improving healthcare services to the patient.

\section*{}
\textbf{Keywords:} Electronic Health Record; Health Information Systems; Interoperability; Information Systems Architecture

\chapter*{Resumo}

As Tecnologias de Informação têm revolucionado as mais diversas áreas da sociedade. Desde há muitos anos atrás, a área da saúde tem procurado soluções tecnológicas que permitam melhorar os processos existentes, tendo como principal objetivo a melhoria dos cuidados de saúde que são prestados. Neste sentido, a disponibilidade da informação clínica sobre um paciente pode ser vital para um diagnóstico e tratamento mais eficazes, por parte de um profissional de saúde. Esta informação deve estar acessível independentemente do contexto, local ou data onde foi coletada. Com o objetivo de providenciar este tipo de dados, muitos países iniciaram projectos de implementação de Registo de Saúde Electrónico (RSE), entre os quais se inclui Portugal, em 2010. 

A complexidade e abrangência faz do RSE um projecto de altos custo e risco. De facto, as dificuldades inerentes a um projecto deste género fazem juz à sua importância e ao impacto que tem nas sociedades. A implementação com sucesso de um RSE é um processo longo e implica, na maioria dos casos, o envolvimento de todos os agentes intervenientes no processo, desde o paciente à instituição de saúde, passando pelos próprios profissionais.

No contexto desta dissertação pretende-se construir uma proposta de arquitectura para implementação do RSE em Portugal. A arquitectura proposta deve servir como uma plataforma de partilha de informação entre as mais diversas entidades de saúde, nor\-ma\-li\-zan\-do os processos e codificações nesta área e promovendo a interoperabilidade entre os agentes. A capacidade de evolução do sistema por um lado, e a facilidade de integração em projectos internacionais por outro, são duas grandes preocupações a ter em conta ao longo do processo de definição da proposta.

O presente documento reflecte o estudo feito no sentido de perceber quais os processos e práticas existentes na implementação destes projectos de larga escala. Nesse sentido, apresentam-se alguns modelos de arquitecturas empresariais assim como alguns estilos de arquitectura, ambos correspondentes a níveis de elevada abstracção. No que diz respeito à integração de diferentes sistemas, são abordadas diversas convenções internacionais para transmissão, codificação e armazenamento da informação clínica, como o HL7, SNOMED CT ou openEHR. Ainda no âmbito deste documento, são analisados dois casos de estudo internacionais do RSE, no Reino Unido e no Canadá. Para concluir, importa sublinhar que, independentemente dos desafios técnicos que se possam apresentar, o processo deve ser sempre guiado pelo objectivo principal de melhorar os cuidados de saúde ao paciente.

\section*{}
\textbf{Palavras-chave:} Registo de Saúde Eletrónico; Sistema de Informação da Saúde; Interoperabilidade; Arquitectura de Sistemas de Informação

%O Resumo fornece ao leitor um sumário do conteúdo da dissertação.
%Deverá ser breve mas conter detalhe suficiente e, uma vez que é a porta
%de entrada para a dissertação, deverá dar ao leitor uma boa impressão
%inicial.
%
%Este texto inicial da dissertação é escrito no fim e resume numa
%página, sem referências externas, o tema e o contexto do trabalho, a
%motivação e os objectivos, as metodologias e técnicas empregues, os
%principais resultados alcançados e as conclusões.
