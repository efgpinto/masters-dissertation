\chapter{Conclusions} \label{chap:concl}

\section*{}

The importance of an Electronic Health Record system and its benefits in the healthcare services equals the difficulty that they face to be implemented. In fact, these kind of platform take many years to be implemented, resulting in high costs. In this context, it is important that this architecture proposal can reflect the existing case studies, in order to avoid approaches that could led to implementation failures. However, the challenges are not only the size of such project. Most of the times, EHR implementations have to deal with and manage multiple external factors, either political, economical or social. These factors tend to limit and condition the execution progress.

At this phase, it is already possible to state some critical success factors in terms of the final architecture proposal. The health area has several actors, multiple institutions and not less business processes. In this scenario, it will be fundamental that the research for designing the proposal can be done close to the main stakeholders, essentially the healthcare organizations IT directors. On the other hand, the Portugal economic situation demands these projects to obtain short-term results. In this dissertation, the main objective is not the short-term perspective but the long-term evolution and vision of the system. However, it is also important that the architecture proposal allows to obtain some quick results, without precluding the future development, not only because of the economic situation but also because that might be the right way to engage all the EHR stakeholders, from patients to healthcare organizations.

The validation of the architecture might be an important point in order to increase confidence and decrease the implementation risks. Despite of that analysis might not be trivial, some evaluation software architecture techniques might be explored and adapted to this context. Due to the project specificities it was essential to close and clearly define the scope of this dissertation, under the penalty of becoming too sparse.

Apart from the conceptual and business challenges, the final purpose must guide all the future work: improving the patient healthcare services allowing the healthcare professionals to access relevant clinical information.
