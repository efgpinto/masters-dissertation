%some macro definitions

% format
\newcommand{\class}[1]{{\normalfont\slshape #1\/}}

% entities
\newcommand{\Feup}{Faculdade de Engenharia da Universidade do Porto}

\newcommand{\svg}{\class{SVG}}
\newcommand{\scada}{\class{SCADA}}
\newcommand{\scadadms}{\class{SCADA/DMS}}


% \myfigure{file}{caption}
\newcommand{\myfigure}[2]{\mycustomfigure{#1}{#2}{width=0.86\textwidth}}

% \mycustomfigure{file}{caption}{params}
\newcommand{\mycustomfigure}[3]{\begin{figure}[t]
  \begin{center}
    \leavevmode
    \includegraphics[#3]{#1}
    \caption{#2}
    \label{fig:#1}
  \end{center}
\end{figure}}

% \mycustomfigure{file}{caption}{params}
\newcommand{\myoriginalfigure}[2]{\begin{figure}[t]
  \begin{center}
    \leavevmode
    \includegraphics{#1}
    \caption{#2}
    \label{fig:#1}
  \end{center}
\end{figure}}

% \myfigure{label}{caption}{table spec}{contents}
\newcommand{\mytable}[4]{\begin{table}[t]
  \footnotesize
  \centering
  \caption{#2}
  \begin{tabular}{#3}
    #4
  \end{tabular}
  \label{tab:#1}
\end{table}}

\newcommand{\myprinciple}[7]{
  \mytable{#1}{#2}{p{2cm}|p{12cm}}{
  \hline Name & #3 \\ 
  \hline Statement & #4 \\ 
  \hline Rationale & #5 \\
  \hline Implications & #6  \\ 
  \hline 
}} 

\newcommand{\todo}[1]{\textbf{\color{red}[TODO #1]}}