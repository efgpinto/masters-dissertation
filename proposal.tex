\chapter{Architecture Proposal} \label{chap:arch-proposal}

\section*{}

\section{Principles}

\subsection{Enterprise principles}

\myprinciple{priciple1}{"Quick-wins strategy" principle}
{Quick-wins strategy}
{The difficult times that Portugal are passing through imply the search for solutions with short-term results and value. In this sense, there is an emergent need for balancing the best solution and the reasonable solution with short-term benefits.}
{The project's scope and complexity obliges the definition and search for solutions that might motivate all the stakeholders. Since there are an huge number of interested parties, the quick-wins are a big help in order to raise interest and contributions to the project. On the other hand, the definition of short-term milestones help to keep control of the progress.}
{}


\myprinciple{principle2}{"Data normalization" principle}
{Data normalization}
{The normalization path seems inevitable. Accordingly, this proposal should privilege and encourage the adoption and evolution of the systems, instead of creating several solutions according with each existing system.}
{The advantages of normalizing data, processes or any other kind of action are recognized and consensual. In fact, with the systems growing up exponentially in number and complexity, the need for sharing data has increased as well, creating a context where a system should be able to communicate and exchange data with thousands of others. The definition of norms for representing the data allows an healthcare institution to much easily receive and interpret data from one other.}
{The change of information systems that are not prepared to be expanded might be very slow and complex, in most cases. Also, there is a need for changing the processes, unifying them, and that change might be more difficult than previous one.}
{Definition of national unified models for documents like discharge letters, analysis and so on.} 


\subsection{IT principles}

\myprinciple{principle3}{"Adoption of international standards" principle} %caption
{Adoption of international standards} %Name
{There are numerous health standards ready to be used and facilitate sharing and exchanging of clinical information. The new services should adopt a standard and the old ones should be converted.} %Statement
{Most of the times, the standards can not be directly applied to the Portuguese reality. That is, the national systems since that they are using national codifications which simply can not be substituted. In that case, the solution might be to create ways of translate between the two codifications. } %Rationale
{Definition of conversion tables or services between the national and international codifications.} %Implications

\subsection{Architectural principles}

\myprinciple{principle4}{"Flexible solutions" principle}
{Flexible solutions}
{With such a long process, there are several decisions taken already in the process of implementation. In that situations, the decisions should consider the solutions that will allow the system to evolute easily.} %statement
{Most of times, the search for flexible solutions have high costs in terms of implementation schedule. Certainly, there is a need for balancing the level of flexibility that the solution might search for. } %rationale
{} %implications

\myprinciple{principle5}{"Fit without precluding" principle}
{Fit without precluding}
{The strategy is not to replace the existing systems but to fit them trying not to compromise future requirements and developments.}
{Most of the systems in use were created several years ago. That means that every healthcare professional is accustomed to the existing systems and that a sudden and unprepared change to the way they have to work and do their job might sentence the project's failure.}
{There will be necessary to understand which systems are fundamental to the business goals, meaning that might be some systems which does not worth to integrate with, either because they are too old and will be deprecated soon or because they simply have not enough audience.}


\section{Structural Concepts}

An Electronic Health Record can be defined as the set of some essential healthcare services. In this section, we will describe the services which we consider the most important ones.

\subsection{Patient Summary}

The Patient Summary (PS)~\citep{EPSOS_PS} is a set of information that allows an healthcare professional to have a quick and easy overview over a patient, used in the epSOS project. A similar concept is used in the England's National Health Service, called Summary Care Records. Although the name may vary, the concept is practically the same and stands as an electronic record that will give healthcare staff faster, easier access to essential information about a patient, to help provide safe treatment in an emergency, generally~\citep{Service2010}.

The epSOS Patient Summary contains the following data:
\begin{itemize}
\item Demographic information (e.g. name, birth date, gender);
\item Most important clinical patient data (e.g. allergies, current medical problems, or major surgical procedures during the last six months);
\item Current medication including all prescribed medicines;
\item Meta-data about the Patient Summary itself (e.g. when and by whom was created or modified).
\end{itemize}

The clinical data included may vary a little bit but, in general terms, is very similar to the set stated above.

\subsubsection{Discussion}

The benefits of the existence of a Patient Summary are not very consensual. In general terms, the advantages announced are:
\begin{itemize}
\item Improved information flow between patient and staff;
\item Better and more accurate treatment in emergency situations;
\item Quicker and easier access to essential information;
\item Easier to allow the patient to check its own PS.
\end{itemize}

In fact, it is possible to point out two completely different case studies, in terms of acceptance and success: England and Scotland. 

In England, several articles~\citep{Anderson2010, Coiera2011, Greenhalgh2010} had been written expressing concerns and doubts about the concept, arguing that the Summary Care Record (in England) should be abandoned "for reasons of safety, functionality, clinical autonomy, patient privacy, and human rights"~\citep{Anderson2010}, stated Ross Anderson. Anderson justifies his critical saying that the data coming from multiple sources, for which there is no responsible, will create a poor and dangerously incomplete summary. Moreover, a final report~\citep{Stramer2010} of an independent evaluation of the Summary Care Record programme concluded that there is limited evidence that the SCR programme had so far achieved the benefits set out. Despite of stating an evidence of improved quality in some consultations, particularly those which involved medication decisions and a probably reduce of rare but important medication errors, the commission was not able neither to find evidence of reduction in onward referral nor to evaluate the impact on the satisfaction of patients. Although, the report defends that the SCR was particularly useful in patients unable to communicate or advocate for themselves. On the other hand, Mark Walport advocates~\citep{Walport2010} that "good information technology has the capacity to be transformational" insisting on the idea of quicker access to more reliable information that should help the treatment.

The Scottish case appears to gather more consensus and positive feedback. A study of Gartner Industry Research stated~\citep{Research2007} that the Emergency Care Summary (ECS), how it is called in Scotland, is a success because it met a specific business need. Also, it advocates that the clinician buy-in was achieved by involving the clinicians in the definition of ECS data and designing the security and access protocol for it as well as there was "training out-of-hours staff to use ECS appropriately" which was identified as a critical factor too. Another study~\citep{Jones2008} on the ECS reported an estimated volume of 5.1 million patient records created from the GP practices and 1.3 million of accesses to them, in 2008.

\subsection{Personal Health Record}

A Personal Health Record is a system whereby individuals can "access, manage and share their health information, and that of others for whom they are authorized, in a private, secure, and confidential environment"~\citep{Health2003}. As it is reported by some articles~\citep{Tang2006,Tang2009}, the approach of the system might differ from being a totally isolated one till being integrated with the national healthcare system. Also, there is a lot of studies~\cite{Tang2009,Pagliari2007,Detmer2008, Fricton2008} about the benefits that those systems can bring:
\begin{itemize}
\item quality, completeness, depth, and accessibility of health information provided by patients;
\item self management support - e.g. care plans, graphing of symptoms, passive biofeedback, tailored instructive or motivational feedback, decision aids, or reminders;
\item communication between patients and providers;
\item access to patients' health knowledge;
\item portability of clinical records and other personal health information;
\item links to static or interactive information about illness, treatments, or self care;
\item capture of symptom or health behaviour data by self report or objective monitoring through electronic devices.
\end{itemize}

This kind of system represents a new paradigm in which the patients have the central power about the information about himself~\citep{Ball2007}. Also, some new visions appeared, extracting the maximum value of the information inserted by the individual. For instance, how that information can be used in a mobile context~\citep{Brief2010}.


\subsection{Professional Portal}
\subsection{epSOS integration}

\section{Information Architecture}
\subsection{Baseline}
\subsection{Ongoing}

The changes in the Information Architecture has not been very significant yet. However, it is important to notice the introduction of the Patient Portal (the PHR system) that centralizes all the patients' information across the country, in the same system.

In a parallel work, there was the need of defining the new data model for the RCU2 (the Patient Summary system). In a team effort with Dr. Raquel Deveza\footnote{Dr. Raquel Deveza - Coordinator of the RCU2 project}, Eng. Miguel Oliveira\footnote{Eng. Miguel Oliveira - FEUP team member}, Dr. José Castanheira\footnote{Dr. José Castanheira - Coordinator of Information Systems Unit of ARSN} it has been defined a database model that would be able to store the RCU2 information. The model can be found at Appendix~\ref{ap1:rcu2_db}.

The model was conceived with two main concerns:
\begin{itemize}
\item Approximate the data model to the existing representation in the healthcare primary care units in order to facilitate the data transfer;
\item Build a RCU2 model based on the epSOS Patient Summary information recommendations and requirements in order to facilitate the data export and dispatch to other countries by epSOS platform.
\end{itemize}

The referenced model will be tried out at the primary healthcare institutions.



\subsection{Future}

As regards to the RCU2, the natural tendency is to group the several institutions by regions and evolving trying to have as few instances as possible. This policy of centralization has great advantages in terms of ease of troubleshooting, improvements, maintenance, etc. 

Even with respect to the RCU2, the possibility of migrating the data model to a standard like HL7 RIM or so must not be discarded. However, it seems impossible in the medium-term future to completely apply a standard to it. There are several reasons to that, but the most prominent one is the effort that it would require to change the existing systems in order to exchange data like that.

\section{Application Architecture}
\subsection{Baseline}
\subsection{Ongoing}
\subsection{Future}


\section{Technology Architecture}
\subsection{Baseline}
\subsection{Ongoing}
\subsection{Future}



