\chapter{Architecture Proposal} \label{chap:arch-proposal}

\section*{}

\section{Principles}

\subsection{Enterprise principles}

\myprinciple{priciple1}{"Quick-wins strategy" principle}
{Quick-wins strategy}
{The difficult times that Portugal are passing through imply the search for solutions with short-term results and value. In this sense, there is an emergent need for balancing the best solution and the reasonable solution with short-term benefits.}
{The project's scope and complexity obliges the definition and search for solutions that might motivate all the stakeholders. Since there are an huge number of interested parties, the quick-wins are a big help in order to raise interest and contributions to the project. On the other hand, the definition of short-term milestones help to keep control of the progress.}
{}


\myprinciple{principle2}{"Data normalization" principle}
{Data normalization}
{The normalization path seems inevitable. Accordingly, this proposal should privilege and encourage the adoption and evolution of the systems, instead of creating several solutions according with each existing system.}
{The advantages of normalizing data, processes or any other kind of action are recognized and consensual. In fact, with the systems growing up exponentially in number and complexity, the need for sharing data has increased as well, creating a context where a system should be able to communicate and exchange data with thousands of others. The definition of norms for representing the data allows an healthcare institution to much easily receive and interpret data from one other.}
{The change of information systems that are not prepared to be expanded might be very slow and complex, in most cases. Also, there is a need for changing the processes, unifying them, and that change might be more difficult than previous one.}
{Definition of national unified models for documents like discharge letters, analysis and so on.} 


\subsection{IT principles}

\myprinciple{principle3}{"Adoption of international standards" principle} %caption
{Adoption of international standards} %Name
{There are numerous health standards ready to be used and facilitate sharing and exchanging of clinical information. The new services should adopt a standard and the old ones should be converted.} %Statement
{Most of the times, the standards can not be directly applied to the Portuguese reality. That is, the national systems since that they are using national codifications which simply can not be substituted. In that case, the solution might be to create ways of translate between the two codifications. } %Rationale
{Definition of conversion tables or services between the national and international codifications.} %Implications

\subsection{Architectural principles}
\myprinciple{principle4}{"Flexible solutions" principle}
{Flexible solutions}
{With such a long process, there are several decisions taken already in the process of implementation. In that situations, the decisions should consider the solutions that will allow the system to evolute easily.} %statement
{Most of times, the search for flexible solutions have high costs in terms of implementation schedule. Certainly, there is a need for balancing the level of flexibility that the solution might search for. } %rationale
{} %implications

\myprinciple{principle5}{"Fit without precluding" principle}
{Fit without precluding}
{The strategy is not to replace the existing systems but to fit them trying not to compromise future requirements and developments.}
{Most of the systems in use were created several years ago. That means that every healthcare professional is accustomed to the existing systems and that a sudden and unprepared change to the way they have to work and do their job might sentence the project's failure.}
{There will be necessary to understand which systems are fundamental to the business goals, meaning that might be some systems which does not worth to integrate with, either because they are too old and will be deprecated soon or because they simply have not enough audience.}


\section{Structural Concepts}

An Electronic Health Record can be defined as the set of some essential healthcare services. In this section, we will describe the services which we consider the most important ones.

\subsection{Patient Summary}

The Patient Summary (PS)~\citep{EPSOS_PS} is a set of information that allows an healthcare professional to have a quick and easy overview over a patient, used in the epSOS project. A similar concept is used in the England's National Health Service, called Summary Care Records. Although the name may vary, the concept is practically the same and stands as an electronic record that will give healthcare staff faster, easier access to essential information about a patient, to help provide safe treatment in an emergency, generally~\citep{Service2010}.

The epSOS Patient Summary contains the following data:
\begin{itemize}
\item Demographic information (e.g. name, birth date, gender);
\item Most important clinical patient data (e.g. allergies, current medical problems, or major surgical procedures during the last six months);
\item Current medication including all prescribed medicines;
\item Meta-data about the Patient Summary itself (e.g. when and by whom was created or modified).
\end{itemize}

The clinical data included may vary a little bit but, in general terms, is very similar to the set stated above.

\subsubsection{Discussion}

The benefits of the existence of a Patient Summary are not very consensual. In general terms, the advantages announced are:
\begin{itemize}
\item Improved information flow between patient and staff;
\item Better and more accurate treatment in emergency situations;
\item Quicker and easier access to essential information;
\item Easier to allow the patient to check its own PS.
\end{itemize}

In fact, it is possible to point out two completely different case studies, in terms of acceptance and success: England and Scotland. 

In England, several articles~\citep{Anderson2010, Coiera2011, Greenhalgh2010} had been written expressing concerns and doubts about the concept, arguing that the Summary Care Record (in England) should be abandoned "for reasons of safety, functionality, clinical autonomy, patient privacy, and human rights"~\citep{Anderson2010}, stated Ross Anderson. Anderson justifies his critical saying that the data coming from multiple sources, for which there is no responsible, will create a poor and dangerously incomplete summary. Moreover, a final report~\citep{Stramer2010} of an independent evaluation of the Summary Care Record programme concluded that there is limited evidence that the SCR programme had so far achieved the benefits set out. Despite of stating an evidence of improved quality in some consultations, particularly those which involved medication decisions and a probably reduce of rare but important medication errors, the commission was not able neither to find evidence of reduction in onward referral nor to evaluate the impact on the satisfaction of patients. Although, the report defends that the SCR was particularly useful in patients unable to communicate or advocate for themselves. On the other hand, Mark Walport advocates~\citep{Walport2010} that "good information technology has the capacity to be transformational" insisting on the idea of quicker access to more reliable information that should help the treatment.

The Scottish case appears to gather more consensus and positive feedback. A study of Gartner Industry Research stated~\citep{Research2007} that the Emergency Care Summary (ECS), how it is called in Scotland, is a success because it met a specific business need. Also, it advocates that the clinician buy-in was achieved by involving the clinicians in the definition of ECS data and designing the security and access protocol for it as well as there was "training out-of-hours staff to use ECS appropriately" which was identified as a critical factor too. Another study~\citep{Jones2008} on the ECS reported an estimated volume of 5.1 million patient records created from the GP practices and 1.3 million of accesses to them, in 2008.

\subsection{Personal Health Record}

A Personal Health Record is a system whereby individuals can "access, manage and share their health information, and that of others for whom they are authorized, in a private, secure, and confidential environment"~\citep{Health2003}. As it is reported by some articles~\citep{Tang2006,Tang2009}, the approach of the system might differ from being a totally isolated one till being integrated with the national healthcare system. Also, there is a lot of studies~\cite{Tang2009,Pagliari2007,Detmer2008, Fricton2008} about the benefits that those systems can bring:
\begin{itemize}
\item quality, completeness, depth, and accessibility of health information provided by patients;
\item self management support - e.g. care plans, graphing of symptoms, passive biofeedback, tailored instructive or motivational feedback, decision aids, or reminders;
\item communication between patients and providers;
\item access to patients' health knowledge;
\item portability of clinical records and other personal health information;
\item links to static or interactive information about illness, treatments, or self care;
\item capture of symptom or health behaviour data by self report or objective monitoring through electronic devices.
\end{itemize}

This kind of system represents a new paradigm in which the patients have the central power about the information about himself~\citep{Ball2007}. Also, some new visions appeared, extracting the maximum value of the information inserted by the individual. For instance, how that information can be used in a mobile context~\citep{Brief2010}.


\subsection{Professional Portal}
\subsection{epSOS integration}









\section{Baseline Architecture}
Before talking about where should we go, we must know the starting point. This section aims to provide a general overview about the current Portuguese Healthcare system, its information systems and the way they integrate or not with each other.

\subsection{Information Architecture}

In order to facilitate this characterization, we will divide the description in two main contexts: the primary healthcare data and the ambulatory healthcare data.

\subsubsection{Primary Healthcare Data}

This kind of data is gathered among the primary healthcare institutions (Cuidados de Saúde Primários - CSP). The WHO defines the primary healthcare with five essential goals: "reducing exclusion and social disparities", "organizing health services around people's needs and expectations", "integrating health into all sectors", "pursuing collaborative models of policy dialogue" and "increasing stakeholder participation". In this sense, the information collected and managed at this level are:
\begin{itemize}
\item Allergies and intolerances;
\item Vaccinations;
\item Resolved, closed or inactive problems;
\item Current problems and diagnosis;
\item Current and past medicines;
\item Eventually some information about the social historic of the patient.
\end{itemize}

As it is possible to observe, nowadays the major problem is not the lack of clinical information but the way that information is (or is not) shared and available when it could make the difference. In fact, this information is created, collected and consumed only in the institution where it belongs.

% %	\subsection{Tertiary Healthcare Data}

\subsubsection{Ambulatory Healthcare Data}

The ambulatory healthcare data is produced at the hospitals and contains data about:
\begin{itemize}
\item External consultation\footnote{External consultation - where the patients, with previous scheduling, are observed, diagnosed and receive therapeutics as well as simple surgical interventions or similar episodes};
\item Vaccinations;
\item Resolved, closed or inactive problems;
\item Current problems and diagnosis;
\item Current and past medicines;
\item Eventually some information about the social historic of the patient.
\end{itemize}

All the public hospitals use a system called Grupos de Diagnósticos Homogéneos (GDH), which are the translation of the  Diagnosis Related Groups (DRG). The GDH is a classification system for hospitalized patients that defines coherent patient groups with a similar estimated cost. Through this system it is possible to state the set of goods and services that each patient receives having in count his needs and the pathology that brought him in. The GDH are used to calculate the financing level that a hospital receives.~\citep{GDH2011}

Moreover, the information is not directly collected or transformed in GDHs. In first place, the information is stored at the local information systems or, in some cases, in paper.~\citep{Ferreira2008} Later, some specialized doctors transform the existing data into the GDH format so the hospital will be able to receive the state's support. Nowadays, the hospitals introduces the GDHs in a national central system called WebGDH. The information dispatch only occurs at the end of each month. The hospital diagnosis and procedures discharge letters are coded using the ICD-9-CM (see Section \ref{sec:icd}).


\textbf{Problems}
The process of transforming the locally stored information into GDH compliant data might modify and discredit it.
The time period of one month to have the data centralized and normalized might be too large.

\subsubsection{Problems}
Have the primary healthcare institutions the capability to host and provide that information? Which uptime and availability levels are we talking about?
The Patient Summary (RCU2) information will be stored where?

The plan is that each primary healthcare institution would be responsible to host the Patient Summary information about their affected patients.

Quais são as estruturas de dados necessárias para suportar os processos ? Qual a sua estrutura lógica e física ? Que tipo de protecção necessita ? Como é organizada, onde está localizada e como alimenta as diversas aplicações que suportam os processos ? Como detectamos e eliminamos as redundâncias ?


\subsection{Application Architecture}


Quais são as aplicações que suportam os processos-chave definidos na Arquitectura do Negócio ? Como interagem entre si ? São centralizados ou descentralizados ? Que informação guardam, como a actualizam e que dependências têm ?

\subsubsection{SONHO}

The SONHO (Sistema Integrado de Informação Hospitalar\footnote{Integrated System for Hospital Information}) is an information system for patient management created in 1988 and disseminated by the Portuguese hospitals, with no competitors at that time.~\citep{Teixeira2005} Nowadays, SONHO is the dominant system in Portuguese hospitals' being installed in about 90\% of the Portuguese public hospitals. SONHO has as the main objective to control the flow of patients in the hospital, that mean to know who is in and who leaves, and what resources were spend with which patient, aiming to ensure some standardization of statistical data and billing. This system started with three modules: outpatient, inpatient and emergency but some new modules were added, like surgery operation room and day care modules.~\citep{Cruz-correia}

This application allows the registration of clinical data (e.g. history of an inpatient encounter, emergency summary report, referral letters, diagnosis and procedures), allowing the establishment of a bridge between the clinical data and the production of the DGHs with billing purposes.~\citep{GDHSONHO2011}

\myfigure{SONHO_ecra1}{First screen of SONHO}

%SONHO is seriously compromised due to its outdated infrastructure, because it de- pends on currently discontinued Oracle Database and Oracle Forms versions. Many hospitals using SONHO are seeking alternatives to it, and thereby facing migration problems.

\subsubsection{SINUS}

The SINUS (Sistema de Informação de Unidades de Saúde\footnote{Information System for Health Institutions}) was created with the same intents of SONHO but for the primary healthcare institution contexts. The SINUS is widely disseminated over Portugal, being present in 3 hospitals and 72 primary health care institutions. Currently, SINUS is still supporting the healthcare unit management, mainly in terms of medical consultation scheduling and vaccination. Furthermore, until that functionalities are replaced, the system must keep operational, although the recommended abandonment of the module "Managing Users".~\citep{Saude2010}


\subsubsection{SAM}

%TODO inserir percentagem de uilização

In 1999, the Ministry of Health decides to start developing a new system called SAM (Sistema de Apoio ao Médico\footnote{Doctor Support System}). The SAM was born as a web software layer, supported upon the SONHO and SINUS systems, aiming to serve as a platform where doctors could introduce information in SINUS and SONHO at the same time. SAM was designing with two essential guide lines~\citep{Castanheira2005, ACSS_SAM2010}:
\begin{itemize}
\item Provide some basic functionalities to the doctor daily activities and related with the data stored at SINUS and SONHO, like scheduling management, reports, few clinical data records, etc.
\item Enable the implementation of some processes which interfere with the doctor activity.
\end{itemize}


\subsubsection{SAPE}
The SAPE~\citep{ACSS_SAPE2010} (Sistema de Apoio à Prática de Enfermagem\footnote{Nursing Practice Support System}) is a software directed to the nursing professionals, that allows the scheduling and recording of the healthcare activities at the health institutions. SAPE was built to address the nurse daily activities, trying to organize and facilitate the information consume. On the other hand, it was also created with the objective of normalizing the nursing records system. The collected data results from the nursing healthcare at the institutions where the system is implemented.

This system uses the CIPE (Classificação Internacional para a Prática de Enfermagem - version BETA 2\footnote{ICNP - International Classification for Nursing Practice of International Council of Nurses}) as a language reference. 


\subsubsection{Problems}

The healthcare professionals have to deal and work with several applications

As aplicações são enumeradas, descritas e classificadas segundo diversos critérios. O seu papel na Arquitectura da Informação é claro e definido, as suas funções têm fronteiras e interfaces expostos ao resto das aplicações como serviços ou componentes reutilizáveis.


\subsection{Technology Architecture}

Primary healthcare institutions with Oracle 7.3.5 databases
One server for each institution
Multiple authentication systems for each information system

Que tecnologias melhor se adaptam às necessidades técnicas ? Que tecnologias têm o melhor retorno de investimento ? Que infra-estruturas de processamento, redes e comunicações são necessárias para suportar as necessidades das aplicações críticas para o negócio ? Quais os níveis de redundâncias físicas necessárias para responder aos níveis de serviço esperados ?







\section{Ongoing Developments}


\subsection{Application Architecture}

The government's vision for the health systems introduced two important portals that we describe in the next subsections.

\subsubsection{The Patient Portal - My Notes}
To the existing Patient Portal was added a new section called "My Notes". This section pretends to be an PHR system. For now, the portal allows an individual to input the following information: emergency contacts, health information (height, weight, blood glucose, or blood pressure), lifestyle, health problems, allergies and medications. As stated at the service, the objective is to allow the citizen to have an active role in managing, promoting and improving its health state.

The system also allows the individual to give or withdraw the consent to share the information among the public healthcare institutions.

\subsubsection{The Professional Portal}

The Professional Portal aims to be the gate through which the doctors will be able to access the clinical information about a patient. In this system, it is intended to gather the information coming from the Patient Portal (introduced by the referenced patient) but also the information stored at other institutions like primary healthcare institutions and hospitals. In this sense, it was necessary to think how it would be possible to integrate the data from two different sources:

Since that the SAM application has a web interface, and it is installed in a significant percentage of the hospitals and primary care institutions, the quicker solution found was to allow the clinicians to directly access the system in order to consult the information. However, to accomplish that, it is necessary that each institutions publishes that service to the public health information network (RIS - Rede de Informação da Saúde). Basically, the work-flow would be something like this:
\begin{enumerate}
\item the clinician at local institution is using SAM to access the local episode and patient information;
\item the clinician clicks in link that takes him to the Professional Portal;
\item the clinician selects one institution accessing directly to the SAM of that institution to consult the information.
\end{enumerate}


\subsubsection{Problems}

In order to achieve this kind of work-flow is necessary to modify: first, the SAM application introducing the link to the Professional Portal; second, the institutions' infrastructure to allow external connections to their SAM system. Since SAM was developed within the Ministry of Health, the first part was relatively easy to accomplish. On the other hand, the second part is much more difficult than initially estimated:
\begin{itemize}
\item it is inconceivable to configure n-to-n permissions so that each institution is able to access all the others. So, the logical solution is to make the request pass through the PDS (Plataforma de Dados da Saúde). With this option, from the target institution point of view the request comes always from PDS, and that is the only external platform to have access to the system;
\item the PDS has to provide a tunnelling service handling the request and the reply, something that can be very heavy when working country-wide.
\end{itemize}

%TODO inserir figura para explicitar o work-flow presente no resumo da reuniao 28022012

\subsection{Information Architecture}
The changes in the Information Architecture has not been very significant yet. However, it is important to notice the introduction of the Patient Portal that centralizes all the patients' information across the country, in the same system.

In a parallel work, there was the need of defining the new data model for the RCU2 (the Patient Summary system). In a team effort with Dr. Raquel Deveza\footnote{Dr. Raquel Deveza - Coordinator of the RCU2 project}, Eng. Miguel Oliveira\footnote{Eng. Miguel Oliveira - FEUP team member}, Dr. José Castanheira\footnote{Dr. José Castanheira - Coordinator of Information Systems Unit of ARSN} it has been defined a database model that would be able to store the RCU2 information. The model can be found at Appendix~\ref{ap1:rcu2_db}.

The model was conceived with two main concerns:
\begin{itemize}
\item Approximate the data model to the existing representation in the healthcare primary care units in order to facilitate the data transfer;
\item Build a RCU2 model based on the epSOS Patient Summary information recommendations and requirements in order to facilitate the data export and dispatch to other countries by epSOS platform.
\end{itemize}

The referenced model will be tried out at the primary healthcare institutions.

\subsection{Technological Architecture}








\section{Future Vision}

The objective of this section is to explain and state some suggestions about what is possible to do in the next two or three years, in order to achieve some kind of interoperability and sharing, with the aim of improving the healthcare services in Portugal. In this sense, the goal is not to predict the future about the health information systems or to propose intangible milestones but to point out some strategies that might help to create value in the mid-term future without precluding future requirements and evolutions.

\subsection{Application Architecture}

In the Section \ref{sec:arch-styles} we referred some architectural styles that could be interesting to this work. When working with complex projects like this, the solution is always an aggregate of smaller solutions. That means that is not possible to simple apply one model or one style but a composition of several. About the styles that we have studied, we can retain some guides and understand what problems they can solve and where we should use them.

The match between the Metropolis Model and a healthcare national system is an almost match and the Portuguese case is not different. In fact, the Portuguese healthcare context is an unstable environment, with many stakeholders and players %TODO explicar melhor o match.
However, the urgency for results prevents the try for a completely fit of the model. Nevertheless, there are several principles that it is possible to extract and apply to the current model. In the next few subsections, we will overview some principles that might be adopted from the studied models.

\subsubsection{Definition of kernel's scope}

The Portuguese EHR project was renamed to \textit{PDS - Plataforma de Dados da Saúde} (Healthcare Data Platform). The recent developments under the sign of PDS has led to the releasing of two systems: the Patient Portal - My Notes and the Professional Portal. However, these two systems were developed almost as independent systems, with their own user interfaces, architecture, technologies and even implemented by different institutions (as represented at Fig. \ref{fig:evolution-phase1}).

\mycustomfigure{evolution-phase1}{The actual situation with the two portals}{scale=0.7}

In order to became a real platform for sharing clinical data, the PDS should evolve with a well delimited scope. Following some principles of the Metropolis Model (see Sec. \ref{sec:metro-model}) it is possible to formulate the next suggestions:
\begin{itemize}
\item delimit the scope of PDS to a platform for sharing data;
\item split the logical domain from the presentation one.
\end{itemize}

The objective is to transform the two applications so that the two existing interfaces became only the official clients to the clinical information repository (PDS), in the case, the official ones. To achieve this, it is necessary to clearly separate the two domains, defining web services that are used to query and retrieve data, like represented at Fig. \ref{fig:evolution-phase2}.

\mycustomfigure{evolution-phase2}{The recommended approach, creating well-separated logic and interface domains}{scale=0.7}

This strategy meets the Metropolis Model principles since we are trying to create a smaller but more robust kernel of clinical information. At the same time, it allows to transform those interfaces into open interfaces. By doing that, many stakeholders would have the opportunity (Fig. \ref{fig:evolution-phase3}) to: 
\begin{itemize}
\item develop new applications using the existing services provided directly by the Ministry;
\item integrate the information in the work-flow of the existing information systems.
\end{itemize}

The most significant part is that, the power and investment of the companies would create significant value to the patients, even with the Ministry spending no money unless the effort to provide and define the services. Sometimes, when talking about crowd-sourced where the information is the fundamental, it is not necessary to do high investment but to provide the right tools to the interested parties.

\mycustomfigure{evolution-phase3}{The possibilities created by opening the interfaces to other stakeholders}{scale=0.7}

\subsubsection{Internal organization}

The PDS is expected to became a really huge system. In that sense, the way it is internally organized is fundamental to allow the addition of new services to fit new requirements. The Service-oriented Architectures (Section \ref{sec:soa}) appear as an interesting solution to solve the scalability and complexity issues.

There are some PDS components that are easily identified as potential services, for instance:
\begin{itemize}
\item Data Anonymizer - receives some data and returns it without being possible to reveal the patient identity;
\item Data Dispatch Manager - receives the data to be transferred to some institution and assures that it arrives the destination;
\item RCU2 Manager - important component to manage all the actions performed that are related with RCU2. It might be an example of a service composed by internal services itself.
\end{itemize}

%TODO falar sobre o epSOS

\subsubsection{Healthcare Professional Identity Provider}

An Healthcare Professional Identity Provider could help to solve some significant problems in the Portuguese health area. In fact, this problem must be divided in two different issues. The first and most important has to do with the creation of a health professionals national record as it exists to the patients, the RNU - Registo Nacional de Utentes (Patients National Record). The aim of the system would be to store all the professionals working in the healthcare services, helping the Ministry and its information system particularly.

The second issue is the creation of an identity provider certifying that one individual is actually an authenticated and authorized healthcare professional and must be able to access some kind of information.

The urgency for this record is recognized amongst the professionals working in the Ministry's information systems. However, despite of the importance that that kind of record might bring to the systems development, it is not recognized by the superior instances since that it is not a project with enough visibility.

\subsection{Information Architecture}

\subsubsection{Where to store the data?}

Regarding the RCU2, the natural tendency is to group the several institutions by regions and evolving trying to have as few instances as possible. This policy of centralization has great advantages in terms of ease of troubleshooting, improvements, maintenance, etc. 


\subsubsection{How to store the data?}

The use of a standard to store clinical information is often the first solution to evaluate. Even with respect to the RCU2, the possibility of migrating the data model to a standard like HL7 RIM, must not be discarded. However, it seems impossible in the medium-term future to completely apply a standard to it. There are several reasons to that, but the most prominent one is the effort that it would require to change the existing systems in order to exchange data like that.
Also, since that the systems are not likely to be substituted in the mid-term future, it is a lot easier to store it in the original format at RCU2. Obviously, when sending that data abroad, it has to be transformed and transcoded to the international standards. Nevertheless, that is a different question of how the information is transferred between institutions, not how it is stored. 


\subsubsection{How to transfer the data?}

The obvious response would be: always with well-known standards. Although, in the chaotic context of several systems and applications not talking with each other that is not a reasonable answer. So, having this scenario into account it is possible to point out these guidelines:
\begin{itemize}
\item the systems in development process or the systems yet to come should be built under some existing international standard;
\item the data that is sent out from the PDS should use standards and interfaces completely defined and clear;
\item the data flow inside the PDS may not be using standards provided that that option brings clearly benefits;
\item 
\end{itemize}


 
\subsection{Technological Architecture}


%TODO falar instituções que não registam electronicamente
%TODO integração de instituiçoes privadas
%TODO estratégia: o que é novo e para fora deve ser com standards


