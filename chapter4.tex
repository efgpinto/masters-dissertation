\chapter{Baseline Architecture} \label{chap:arch-proposal}

\section*{}

Before talking about where should we go, we must know the starting point. This section aims to provide a general overview about the current Portuguese Healthcare system, its information systems and the way they integrate or not with each other.

\section{Information Architecture}

In order to facilitate this characterization, we will divide the description in two main contexts: the primary healthcare data and the ambulatory healthcare data.

\subsection{Primary Healthcare Data}

This kind of data is gathered among the primary healthcare institutions (Cuidados de Saúde Primários - CSP). The WHO defines the primary healthcare with five essential goals: "reducing exclusion and social disparities", "organizing health services around people's needs and expectations", "integrating health into all sectors", "pursuing collaborative models of policy dialogue" and "increasing stakeholder participation". In this sense, the information collected and managed at this level are:
\begin{itemize}
\item Allergies and intolerances;
\item Vaccinations;
\item Resolved, closed or inactive problems;
\item Current problems and diagnosis;
\item Current and past medicines;
\item Eventually some information about the social historic of the patient.
\end{itemize}

As it is possible to observe, nowadays the major problem is not the lack of clinical information but the way that information is (or is not) shared and available when it could make the difference. In fact, this information is created, collected and consumed only in the institution where it belongs.

% %	\subsection{Tertiary Healthcare Data}

\subsection{Ambulatory Healthcare Data}

The ambulatory healthcare data is produced at the hospitals and contains data about:
\begin{itemize}
\item External consultation\footnote{External consultation - where the patients, with previous scheduling, are observed, diagnosed and receive therapeutics as well as simple surgical interventions or similar episodes};
\item Vaccinations;
\item Resolved, closed or inactive problems;
\item Current problems and diagnosis;
\item Current and past medicines;
\item Eventually some information about the social historic of the patient.
\end{itemize}

All the public hospitals use a system called Grupos de Diagnósticos Homogéneos (GDH), which are the translation of the  Diagnosis Related Groups (DRG). The GDH is a classification system for hospitalized patients that defines coherent patient groups with a similar estimated cost. Through this system it is possible to state the set of goods and services that each patient receives having in count his needs and the pathology that brought him in. The GDH are used to calculate the financing level that a hospital receives.~\citep{GDH2011}

Moreover, the information is not directly collected or transformed in GDHs. In first place, the information is stored at the local information systems or, in some cases, in paper.~\citep{Ferreira2008} Later, some specialized doctors transform the existing data into the GDH format so the hospital will be able to receive the state's support. Nowadays, the hospitals introduces the GDHs in a national central system called WebGDH. The information dispatch only occurs at the end of each month. The hospital diagnosis and procedures discharge letters are coded using the ICD-9-CM (see Section \ref{sec:icd}).


\textbf{Problems}
The process of transforming the locally stored information into GDH compliant data might modify and discredit it.
The time period of one month to have the data centralized and normalized might be too large.

\subsection{Problems}
Have the primary healthcare institutions the capability to host and provide that information? Which uptime and availability levels are we talking about?
The Patient Summary (RCU2) information will be stored where?

The plan is that each primary healthcare institution would be responsible to host the Patient Summary information about their affected patients.

Quais são as estruturas de dados necessárias para suportar os processos ? Qual a sua estrutura lógica e física ? Que tipo de protecção necessita ? Como é organizada, onde está localizada e como alimenta as diversas aplicações que suportam os processos ? Como detectamos e eliminamos as redundâncias ?


\section{Application Architecture}


Quais são as aplicações que suportam os processos-chave definidos na Arquitectura do Negócio ? Como interagem entre si ? São centralizados ou descentralizados ? Que informação guardam, como a actualizam e que dependências têm ?

\subsection{SONHO}

The SONHO (Sistema Integrado de Informação Hospitalar\footnote{Integrated System for Hospital Information}) is an information system for patient management created in 1988 and disseminated by the Portuguese hospitals, with no competitors at that time.~\citep{Teixeira2005} Nowadays, SONHO is the dominant system in Portuguese hospitals' being installed in about 90\% of the Portuguese public hospitals. SONHO has as the main objective to control the flow of patients in the hospital, that mean to know who is in and who leaves, and what resources were spend with which patient, aiming to ensure some standardization of statistical data and billing. This system started with three modules: outpatient, inpatient and emergency but some new modules were added, like surgery operation room and day care modules.~\citep{Cruz-correia}

This application allows the registration of clinical data (e.g. history of an inpatient encounter, emergency summary report, referral letters, diagnosis and procedures), allowing the establishment of a bridge between the clinical data and the production of the DGHs with billing purposes.~\citep{GDHSONHO2011}

\myfigure{SONHO_ecra1}{First screen of SONHO}

%SONHO is seriously compromised due to its outdated infrastructure, because it de- pends on currently discontinued Oracle Database and Oracle Forms versions. Many hospitals using SONHO are seeking alternatives to it, and thereby facing migration problems.

\subsection{SINUS}

The SINUS (Sistema de Informação de Unidades de Saúde\footnote{Information System for Health Institutions}) was created with the same intents of SONHO but for the primary healthcare institution contexts. The SINUS is widely disseminated over Portugal, being present in 3 hospitals and 72 primary health care institutions. Currently, SINUS is still supporting the healthcare unit management, mainly in terms of medical consultation scheduling and vaccination. Furthermore, until that functionalities are replaced, the system must keep operational, although the recommended abandonment of the module "Managing Users".~\citep{Saude2010}


\subsection{SAM}

In 1999, the Ministry of Health decides to start developing a new system called SAM (Sistema de Apoio ao Médico\footnote{Doctor Support System}). The SAM was born as a web software layer, supported upon the SONHO and SINUS systems, aiming to serve as a platform where doctors could introduce information in SINUS and SONHO at the same time. SAM was designing with two essential guide lines~\citep{Castanheira2005, ACSS_SAM2010}:
\begin{itemize}
\item Provide some basic functionalities to the doctor daily activities and related with the data stored at SINUS and SONHO, like scheduling management, reports, few clinical data records, etc.
\item Enable the implementation of some processes which interfere with the doctor activity.
\end{itemize}


\subsection{SAPE}
The SAPE~\citep{ACSS_SAPE2010} (Sistema de Apoio à Prática de Enfermagem\footnote{Nursing Practice Support System}) is a software directed to the nursing professionals, that allows the scheduling and recording of the healthcare activities at the health institutions. SAPE was built to address the nurse daily activities, trying to organize and facilitate the information consume. On the other hand, it was also created with the objective of normalizing the nursing records system. The collected data results from the nursing healthcare at the institutions where the system is implemented.

This system uses the CIPE (Classificação Internacional para a Prática de Enfermagem - version BETA 2\footnote{ICNP - International Classification for Nursing Practice of International Council of Nurses}) as a language reference. 


\subsection{Problems}

The healthcare professionals have to deal and work with several applications

As aplicações são enumeradas, descritas e classificadas segundo diversos critérios. O seu papel na Arquitectura da Informação é claro e definido, as suas funções têm fronteiras e interfaces expostos ao resto das aplicações como serviços ou componentes reutilizáveis.


\section{Technology Architecture}

Primary healthcare institutions with Oracle 7.3.5 databases
One server for each institution
Multiple authentication systems for each information system

Que tecnologias melhor se adaptam às necessidades técnicas ? Que tecnologias têm o melhor retorno de investimento ? Que infra-estruturas de processamento, redes e comunicações são necessárias para suportar as necessidades das aplicações críticas para o negócio ? Quais os níveis de redundâncias físicas necessárias para responder aos níveis de serviço esperados ?
